%\documentclass[draft,10pt]{beamer}
\documentclass[10pt]{beamer}

\usepackage[english,polish]{babel}
\usepackage[utf8]{inputenc}
\usepackage{polski}

\usepackage{define}
\usepackage{tikz-uml}

\eventtitle{Programowanie obiektowe w języku C++}
\title{Programowanie obiektowe w języku C++}
\author[shortname]{Stanis{\l}aw Gepner}
\institute[shortinst]{sgepner@meil.pw.edu.pl}
\date{}

\setbeamertemplate{blocks}[rounded][shadow=true]
\setbeamertemplate{navigation symbols}[]

\begin{document}

\frame{
    \titlepage
}

\section{Dziedziczenie - początek}

\begin{frame}
  \frametitle{Dziedziczenie}
  \framesubtitle{Wstęp}

	\centering
	Zacznijmy od przykładu, w którym dziedziczenie by się przydało, ale go jeszcze nie użyjemy.
  
  \begin{itemize}
    \item Przykład rolniczy
    \item Każde zwierzątko wydaje dźwięk: {\it metoda dźwięk()}
    \item Każde się tak samo porusza: {\it metoda chodź()}
    \item Nasze klasy w żaden sposób nie są ze sobą powiązane, musimy traktować je oddzielnie, co jest dość kłopotliwe
    \item Nie możemy na problem spojrzeć abstrakcyjnie
		\item Widzimy, że przydał by się mechanizm, który pozwoli nam traktować zwierzątka na farmie {\it abstrakcyjnie}*
  \end{itemize}
  
  \vspace*{1cm}
  *{\tiny W trakcie przygotowywania wykładu nie zostało skrzywdzone żadne zwierze.}
\end{frame}

\begin{frame}
  \frametitle{Dziedziczenie}
  \framesubtitle{Wstęp}
  	\begin{itemize}
	    \item Kolejne klasy rozszerzają możliwości już istniejących
	    \item Pozwala tworzyć nowe klasy w oparciu o już istniejące
	    \begin{itemize}
	    	\item Współdzielenie kodu (nasza metoda chodz())
	    	\item Łatwiejsze zarządzanie
	    \end{itemize}
	    \item Umożliwia hierarchię klas, np.: kształty $\varsupsetneq$ prostokąty $\varsupsetneq$ kwadraty
	    \item Klasę po której dziedziczymy nazywamy \textbf{bazową}
	    \item Klasę dziedziczącą nazywamy \textbf{pochodną}
	    \item Obiekt klasy pochodnej, jest jednocześnie obiektem klasy bazowej, ale nie odwrotnie
	    \item albo obiekt klasy bazowej jest subobiektem klasy pochodnej (Pochodna ma w sobie bazowy) 
  \end{itemize}
\end{frame}

\section{Składnia}

\begin{frame}[fragile]
  \frametitle{Składnia}
  \begin{center}
\begin{lstlisting}
class nazwa :[operator_widocznosci] nazwa_klasy_bazowej //na razie
{
  //definicja_klasy
};
\end{lstlisting}
  Operator widoczności może być: \textit{public, \textbf{protected}, private}\\ \vspace{0.5cm}

  \textbf{B} dziedziczy po \textbf{A}  
\begin{lstlisting}
class A {};
class B :public A {}; //operator public
\end{lstlisting}
  \centering
  \begin{tikzpicture}
    \umlsimpleclass{A}
    \umlsimpleclass[x=4, y=0]{B}
    \umlinherit{B}{A}
  \end{tikzpicture}
\end{center}
\end{frame}

\begin{frame}[fragile]
  \frametitle{Składnia}
  \framesubtitle{\textbf{\textit{protected}}}
  \centering
  \vspace{-0.5cm}
  Operator widoczności \textit{\textbf{protected}} powoduje, że atrybuty klasy bazowej,
  mogą być swobodnie używane w klasie dziedziczącej ale poza klasą nią
  i klasą bazową nie są widoczne (jak \textit{private}).
  \newline
\begin{lstlisting}
class A{
  public:
    int a;
  protected:
    int b;
  private:
    int c;
};
class B: public A{
  public:
  void fun(){a=8; b=4; c=5;} // c jest private!
};
int main(){
  B b;
  b.fun();
  cout << b.a << " " << b.b << " " << b.c << endl;
}
\end{lstlisting}
\end{frame}

\begin{frame}
  \frametitle{Składnia}
  \framesubtitle{Operator widoczności}
  \begin{itemize}
    \item \textit{public} $\rightarrow$ bez zmian
    \begin{itemize}
      \item \textit{public} $\rightarrow$ \textit{public}
      \item \textit{protected} $\rightarrow$ \textit{protected}
      \item \textit{private} $\rightarrow$ brak dostępu
    \end{itemize}
    \item \textit{protected} $\rightarrow$ \textit{public} zmienia się w \textit{protected}
    \begin{itemize}
      \item \textit{public} $\rightarrow$ \textit{\textbf{protected}}
      \item \textit{protected} $\rightarrow$ \textit{protected}
      \item \textit{private} $\rightarrow$ brak dostępu
    \end{itemize}
    \item \textit{private} $\rightarrow$ wszystko zmienia się w \textit{private}
    \begin{itemize}
      \item \textit{public} $\rightarrow$ \textit{\textbf{private}}
      \item \textit{protected} $\rightarrow$ \textit{\textbf{private}}
      \item \textit{private} $\rightarrow$brak dostępu
    \end{itemize}
    \item brak operatora $\rightarrow$ domyślnie \textit{private}
  \end{itemize}
\end{frame}

\section{Pewne konsekwencje}

\begin{frame}[fragile]
  \frametitle{sizeof}
  \framesubtitle{Tablice}
  \begin{columns}
    \begin{column}{0.49\textwidth}
\begin{lstlisting}
class A{
  public:
  char tabA[1024];
  int a;
};
class B : public A{
  public:
  char tabB[1024];
  int b;
};
...
A a;
B b;
cout<<sizeof(a)<<" "<<sizeof(b)<<endl;
A tab[2];
tab[0] = B();
tab[1] = A(); //zadziala, ale ...
...
tab[0].a = 5;
tab[0].b = 5; //?
tab[1].a = 6;
\end{lstlisting}
    \end{column}
    \begin{column}{0.49\textwidth}
    \begin{itemize}
  \item Obiekt klasy pochodnej zawiera w sobie subobiekt klasy bazowej
  \item Jaka jest odległość między tab[0] i tab[1] ?
  \item Czy zmieści sie tam obiekt klasy B?
  \item Należy uzywać tablicy wskaźników!
\end{itemize}
    \end{column}
  \end{columns}
\end{frame}

\begin{frame}[fragile]
  \frametitle{Dostęp do atrybutów}
  \begin{columns}
    \begin{column}{0.49\textwidth}
      \begin{lstlisting}
class Bydle{
  public:
  void jedz();
};
class kura : public bydle{
  public:
  void gdacz();
};
class krowa : public bydle{
  public:
  void mucz();
};
...
kura k; //instancja
k.jedz()
k.gdacz()

krowa * pk = new krowa();
pk->jedz(); // pointer
pk->mucz();

bydle* pb = pk;//rzutowanie
pk->jedz();
pb = &k;
k->jedz();
((bydle)k).jedz();
\end{lstlisting}
    \end{column}
    \begin{column}{0.49\textwidth}
    \begin{itemize}
  \item Krowa muczy, kura gdzcze
  \item Kura i krowa mogą jeść
  \item przykład ...
\end{itemize}
\begin{lstlisting}
bydle* tab[2];
tab[0] = new kura();
tab[1] = new krowa();
tab[0]->jedz();
tab[1]->jedz();
...
\end{lstlisting}
    \end{column}
  \end{columns}
\end{frame}

\section{Wielodziedziczenie}
\begin{frame}[fragile]
  \frametitle{Wielodziedziczenie}
  \framesubtitle{Diament}
  \begin{columns}
    \begin{column}{0.49\textwidth}
    \begin{tikzpicture}
    \umlsimpleclass{A}
    \umlsimpleclass[x=-1, y=-2]{B}
    \umlsimpleclass[x=1,y=-2]{C}
    \umlsimpleclass[y=-4]{D}
    \umlinherit{B}{A}
    \umlinherit{C}{A}
    \umlinherit{D}{B}
    \umlinherit{D}{C}
    \end{tikzpicture}
    \end{column}
    \begin{column}{0.49\textwidth}
    \begin{itemize}
      \item Język C++ umożliwia wielodziedziczenie
      \item Może to powodować pewne komplikacje - diament
      \item Jak D zawiera A?
      \item Jaki jest rozmiar D?
      \item Jak jednoznacznie dostać się do atrybutów A
      \item Dziedziczenie wirtualne
    \end{itemize}
\begin{lstlisting}
class A {
    char buff[1024];
public:
    void show() {}
};
class B: public A {};
class C: public A {};
class D: public A, public B {};
\end{lstlisting}
    \end{column}
  \end{columns}

\end{frame}

\begin{frame}[fragile]
  \frametitle{Wielodziedziczenie}
  \framesubtitle{Dziedziczenie wirtualne}
  \begin{columns}
    \begin{column}{0.49\textwidth}
    \begin{tikzpicture}
    \umlsimpleclass{A}
    \umlsimpleclass[x=-1, y=-2]{B}
    \umlsimpleclass[x=1,y=-2]{C}
    \umlsimpleclass[y=-4]{D}
    \umlinherit{B}{A}
    \umlinherit{C}{A}
    \umlinherit{D}{B}
    \umlinherit{D}{C}
    \end{tikzpicture}
    \end{column}
    \begin{column}{0.49\textwidth}
    \begin{lstlisting}
class A {
    char buff[1024];
public:
    void show() {}
};
class B: public virtual A {};
class C: public virtual A {};
class D: public A, public B {};
\end{lstlisting}
    \begin{itemize}
      \item D w jednoznaczny sposób zawiera A
      \item D ma odpowiedni rozmiar
      \item Odwołania do atrybutów A są jednoznaczne
    \end{itemize}
    \end{column}
  \end{columns}
\end{frame}

\section{Przykrywanie metod}

\begin{frame}[fragile]
  \frametitle{Przykrywanie metod}
  \begin{columns}
    \begin{column}{0.49\textwidth}
      \begin{lstlisting}
class Bydle{
  public:
  void jedz();
  void dzwiek(){/*?*/}
};
class kura : public bydle{
  public:
  void dzwiek(){cout<<"KoKoKo"<<endl;}
//  void gdacz();
};
class krowa : public bydle{
  public:
  void dzwiek(){cout<<"Muuu"<<endl;}
//  void mucz();
};
...
kura k; k.dzwiek()
krowa a; a.dzwiek;
bydle *p = new krowa();
p->dzwiek(); //??
\end{lstlisting}
    \end{column}
    \begin{column}{0.49\textwidth}
\begin{lstlisting}
bydle* tab[2];
tab[0] = new kura();
tab[1] = new krowa();
tab[0]->dzwiek();
tab[1]->dzwiek();
...
\end{lstlisting}
    \begin{itemize}
  \item Zaczynamy unifikować nasze podejście
  \item Nie interesuje nas czym dokładnie jest bydle
  \item Co jest wywołane, kiedy i dlaczego?
  \item Przykład
\end{itemize}
    \end{column}
  \end{columns}
\end{frame}

\begin{frame}[fragile]
  \frametitle{Funkcje wirtualne}
  \begin{columns}
    \begin{column}{0.49\textwidth}
      \begin{lstlisting}
class Bydle{
  public:
  void jedz();
  virtual void dzwiek(){/*?*/}
};
class kura : public bydle{
  public:
  virtual void dzwiek()
  {cout<<"KoKoKo"<<endl;}
//  void gdacz();
};
class krowa : public bydle{
  public:
  virtual void dzwiek()
  {cout<<"Muuu"<<endl;}
//  void mucz();
};
...
kura k; k.dzwiek()
krowa a; a.dzwiek;
bydle *p = new krowa();
p->dzwiek(); //??
\end{lstlisting}
    \end{column}
    \begin{column}{0.49\textwidth}
\begin{lstlisting}
bydle* tab[2];
tab[0] = new kura();
tab[1] = new krowa();
tab[0]->dzwiek();
tab[1]->dzwiek();
...
\end{lstlisting}
    \begin{itemize}
  \item Pozwala na wywołanie prawidłowej metody
  \item tzn. metody klasy pochodnej z poziomu bazowej
  \item Funkcja nie jest dowiązana w czasie kompilacji, a w momencie wykonania programu - \textit{late linking}
  \item Koszt pamięciowy i czasowy ... nie jeżeli dostajemy się rzutując na obiekt klasy pochodnej
\end{itemize}
    \end{column}
  \end{columns}
\end{frame}

\begin{frame}[fragile]
  \frametitle{Konstrukto Destruktor}
  \framesubtitle{Kolejność wywoływania}
  \begin{columns}
    \begin{column}{0.49\textwidth}
      \begin{lstlisting}
class A{
  public:
  A(){cout<<"A"<<endl;}
  A(int a){...}
  virtual ~A(){}
};
class B : public A{
  public:
  B() : A() {cout<<"B"<<endl;}
  B(int a) : A(a) {...}
  virtual ~B(){}
};
...
A a, a2(9);
B b, b1(2);
...
\end{lstlisting}
    \end{column}
    \begin{column}{0.6\textwidth}
    \begin{itemize}
  \item Subobiekt klasy bazowej musi istnieć zanim powstanie obiekt dziedziczący
  \item Konstruktor obiektu klasy bazowej musi zostać wywołany w liście inicjalizacyjnej
  \item Destruktory powinny być wirtualne, by zapewnić poprawne zwalnianie pamięci
  \item Częste pytanie na rozmowach o pracę: Po co destruktor jest wirtualny?
  \item Konstruktor wywoływany jest od bazowej w dół
  \item Destruktor od pochodnej w górę (chyba, że nie jest wirtualny)
\end{itemize}
    \end{column}
  \end{columns}
\end{frame}

\section{Metody i klasy abstrakcyjne}

\begin{frame}[fragile]
  \frametitle{Metoda abstrakcyjna}
  \begin{columns}
    \begin{column}{0.49\textwidth}
      \begin{lstlisting}
class A{
  public:
  virtual void fun() = 0;
};
class B : public A{
  public:
  virtual void fun(){}
};
\end{lstlisting}
    \end{column}
    \begin{column}{0.6\textwidth}
    \begin{itemize}
  \item Metoda nie musi posiadać ciała w klasie bazowej
  \item ...bo możemy nie wiedzieć jak ją zrealizować
  \item ...stosunek ovwodu do pola w kształcie?
  \item deklaracja = 0;
  \item Można wywołać taką metodę w klasie bazowej!
  \item Klasa posiadająca przynajmniej jedna metodę \textit{abstrakcyjną} staje się klasą \textit{abstrakcyjną} i nie można stworzyć już obiektu tej klasy!
  \item Trzeba dziedziczyć i implementować wszystkie metody abstrakcyjne
\end{itemize}
    \end{column}
  \end{columns}
\end{frame}

\section{Uniemożliwienie dziedziczenia}

\begin{frame}[fragile]
  \frametitle{Uniemożliwienie dziedziczenia}
  \begin{columns}
    \begin{column}{0.6\textwidth}
    \begin{itemize}
  \item Przed C++11 kłopotliwe
  \item Prywatny konstruktor
  \item Dodać do klasy statyczną metodę tworzącą instancję
  \item Nieczytelne
  \item Dla każdego konstruktora musi byc metoda
  \item Kompilator zgłosi błąd dopiero przy instantacji klasy
\end{itemize}
    \end{column}
    \begin{column}{0.47\textwidth}
      \begin{lstlisting}
class A{
private:
  A();
public:
  static A* zrobA() {return new A();}
};
class B: public A{};
int main(){
  // B b; //konstruktor prywatny
  A* pa = A::zrobA();
  delete pa;
  return 0;
}

\end{lstlisting}
    \end{column}
  \end{columns}
\end{frame}

\begin{frame}[fragile]
  \frametitle{Uniemożliwienie dziedziczenia}
  \begin{columns}
    \begin{column}{0.6\textwidth}
    \begin{itemize}
  \item C++11 wprowadza słówko \textit{final}
  \item jak w Java?
\end{itemize}
    \end{column}
    \begin{column}{0.47\textwidth}
      \begin{lstlisting}
class A final{
public:
  A();
};
class B: public A{};
int main(){
  
}
\end{lstlisting}
    \end{column}
  \end{columns}
\end{frame}

\section{Lepsza farma}

\begin{frame}
  \frametitle{Lepsza farma}
  \framesubtitle{Przykład}

	\centering
	Zakończymy przykładem\\
	
	\vspace{1cm}
\begin{tikzpicture}
  \umlsimpleclass{bydle}
  \umlsimpleclass[x=-4, y=-2]{krowa}
  \umlsimpleclass[x=-2, y=-2]{kon}
  \umlsimpleclass[x=0, y=-2]{osiol}
  \umlsimpleclass[x=2, y=-2]{kura}
  \umlsimpleclass[x=-1, y=-4]{mul}
  \umlinherit{krowa}{bydle}
  \umlinherit{kon}{bydle}
  \umlinherit{osiol}{bydle}
  \umlinherit{kura}{bydle}
  \umlinherit{mul}{kon}
  \umlinherit{mul}{osiol}
\end{tikzpicture}
\end{frame}

\end{document}
