%\documentclass[draft,10pt]{beamer}
\documentclass[10pt]{beamer}

\usepackage[english,polish]{babel}
\usepackage[utf8]{inputenc}
\usepackage{polski}

\usepackage{define}

\eventtitle{Programowanie obiektowe w języku C++}
\title{Programowanie obiektowe w języku C++}
\author[shortname]{Stanis{\l}aw Gepner}
\institute[shortinst]{sgepner@meil.pw.edu.pl}
\date{}

\setbeamertemplate{blocks}[rounded][shadow=true]
\setbeamertemplate{navigation symbols}[]

\begin{document}

\frame{
    \titlepage
}

\section{STL}

\begin{frame}
  \frametitle{STL}
  \framesubtitle{Standard Template Library}
  
  \begin{itemize}
    \item Biblioteka szablonów - generyczna
    \item Kontenery (wektor, mapa, lista, ...)
    \item Z typami wbudowanymi oraz programisty
    \item Iteratory - służą do poruszania się po kolekcjach
    \item ... trochę jak wskaźniki, można inkrementować przez \textit{++},
    wyłuskać przez \textit{*} i porównywać przez \textit{!=} 
    \item Algorytmy - czyli wzorce funkcji
    \item STL jest bardzo popularny
  \end{itemize}
\end{frame}

\begin{frame}[fragile]
  \frametitle{Iteratory}
  
\begin{lstlisting}
std::nazwa_kolekcji<parametr template>::iterator nazwa iteratora
\end{lstlisting}
  
  \begin{itemize}
    \item Pozwalają na generyczność STL
    \item Sposób na dostęp do elementów przechowywanych przez kontener
    \item W zależności od kontenera mogą być różne
    \begin{itemize}
      \item input iterator: pozwala na odczyt, nie na zapis
      \item output iterator: pozwala na zapis
      \item forward iterator: odczyt, zapis, tylko ruch do przodu, od początku do końca
      \item bidirectional iterator: także do tyłu
      \item random access: dowolny dostęp
      \item np. vector posiada random access, a lista tylko bidirectional
      \item to znaczy, ze możemy przesuwać się tylko o jeden element
    \end{itemize}
  \end{itemize}
\end{frame}

\section{Kontenery}

\begin{frame}
  \frametitle{Kontenery}
  \begin{itemize}
    \item Obiekty do przechowywania danych
    \item Sekwencyjne: wektor, lista, ...
    \item Asocjacyjne - przechowuje pary (klucz, wartość), umożliwia dostęp przez podanie klucza:
    set, map ...
    \item Też: kolejka (queue) i stack
  \end{itemize}
\end{frame}

\subsection{Wektor - vector}

\begin{frame}[fragile]
  \frametitle{vector}
  \framesubtitle{Tablica dynamiczna}
  \begin{columns}
    \begin{column}{0.6\textwidth}
    \begin{itemize}
      \item Jak tablica dynamiczna w C - ciągły obszar
      \item Możliwy dostęp do dowolnego elementu w stałym czasie
      \item Może automatycznie zmienić rozmiar przy dodawaniu / usuwaniu elementu
      \item Dodawanie / usuwanie na końcu - stały czas
      \item Może zaistnieć konieczność przeniesienia
      \item Dodawanie / usuwanie 'w środku' - czas liniowy bo trzeba przenieść elementy
      \item Dobry jeżeli dodajemy tylko na końcu, znamy rozmiar, często potrzebny dostęp do elementów
    \end{itemize}
    \end{column}
    \begin{column}{0.47\textwidth}
\begin{lstlisting}
#include <vector>

std::vector<typ_przechowywany> nazwa; //pusty
std::vector<typ_przechowywany> nazwa2; = {a1,a2,a3};//3 elementy
std::vector<int> nazwa3(4,100); // 4 int o wartosci 100
std::vector<int> nazwa4(third); // kopia nazwa3

//ale tez:
int myints[] = {16,2,77,29};
std::vector<int> nazwa5 (myints, myints + 4 );

//dostep przez []
cout << nazwa5[3] << endl;
\end{lstlisting}
    \end{column}
  \end{columns}
\end{frame}

\begin{frame}[fragile]
  \frametitle{Vector}
  \framesubtitle{Dostęp}
  \begin{itemize}
    \item {[]} dostęp do elementu
    \item front() - pierwszy element
    \item back() - ostatni element
  \end{itemize}
  \begin{lstlisting}
#include <iostream>
#include <vector>

int main ()
{
  std::vector<int> vec(5, 1); // 5 jedynek
  vec.front() = 9;
  vec.back() = 2;

  std::cout << "vec contains:";
  for (unsigned i=0; i<vec.size() ; i++)
    std::cout << ' ' << vec[i];
  std::cout << '\n';

  return 0;
}
  \end{lstlisting}
\end{frame}

\begin{frame}[fragile]
  \frametitle{Vector}
  \framesubtitle{Iteratory}
  \begin{itemize}
    \item Dowolny dostęp
    \item Bidirectional
    \item \textit{begin()} - iterator do pierwszego elementu
    \item \textit{end()} - iterator za ostatni element
  \end{itemize}
  \begin{lstlisting}
#include <iostream>
#include <vector>

int main ()
{
  std::vector<int> moj_vec = {16,2,77,29}

  std::cout << "moj_vec:";
  for (std::vector<int>::iterator it = moj_vec.begin() ; it != moj_vec.end(); ++it)
    std::cout << ' ' << *it;
  std::cout << '\n';

  return 0;
}
\end{lstlisting}
\end{frame}

\begin{frame}[fragile]
  \frametitle{Vector}
  \framesubtitle{Rozmiar}
  \begin{itemize}
    \item Wektor zajmuje więcej pamięci niż konieczne, by uniknąć realokacji
    \item \textit{size} - bieżący rozmiar,
    \item \textit{max size} - rozmiar maksymalny
    \item \textit{capacity} - Za alokowany rozmiar, niekoniecznie \textit{size}
    \item \textit{empty} - test czy pusty
    \item \textit{reserve} - zarezerwój rozmiar, zmienia \textit{capacity}.
    Jeżeli wystąpi realokacja koszt najwyżej liniowy z rozmiarem
    \item \textit{resize} - Zmienia rozmiar. Liniowy z liczbą dodawanych/usuwanych elementów,
    jeżeli realokacja koszt liniowy z rozmiarem
  \end{itemize}
  \begin{lstlisting}
#include <iostream>
#include <vector>

int main (){
  std::vector<int> vec;
  vec.resize(100, 0);
  for (int i=0; i<100; i++) vec[i]=i;;

  std::cout << "size: " << (int) vec.size() << '\n';
  std::cout << "capacity: " << (int) vec.capacity() << '\n';
  std::cout << "max_size: " << (int) vec.max_size() << '\n';
  return 0;
}
\end{lstlisting}
\end{frame}

\begin{frame}[fragile]
  \frametitle{Vector}
  \framesubtitle{Zmiany}
  \begin{itemize}
    \item \textit{push back} - dodaje element na końcu. koszt stały chyba, że realokacja
    \item \textit{pop back} - Kasuje ostatni. Stały
    \item \textit{insert} - wstawia element - Koszt liniowy z liczbą wstawionych elementów plus
    liczba elementów po - przesuwanych 
    \item \textit{erase} - usuwa elementy. Koszt liniowy z liczną usuwanych plus przesunięcia
    \item \textit{swap} - wymienia zawartość z innym. Stały
    \item \textit{clear} - czyści zawartość. Liniowy
  \end{itemize}
  \begin{lstlisting}
#include <iostream>
#include <vector>
int main (){
  std::vector<int> vec;
  int val=10;
  do {
    vec.push_back (val);
    --val;
  } while (val);
  std::cout << "vec stores " << vec.size() << " numbers.\n";
  return 0;
}
\end{lstlisting}
\end{frame}

\subsection{Lista - list}

\begin{frame}[fragile]
  \frametitle{list}
  \framesubtitle{}
  \begin{columns}
    \begin{column}{0.6\textwidth}
    \begin{itemize}
      \item Pozwala na szybkie dodawanie usuwanie elementów
      \item Nie koniecznie ciągła w pamięci
      \item Elementy w rożnych miejscach
      \item Brak dostępu bezpośredniego
      \item Np. Dostęp do 7 elementu wymaga przejści przez 6 poprzednich
      \item Wymaga dodatkowej informacji o sąsiadach - więcej pamięci
      \item Dobre do częstego wstawiania, usuwania, przenoszenia
    \end{itemize}
    \end{column}
    \begin{column}{0.47\textwidth}
\begin{lstlisting}
#include <list>

std::list<int> first;//pusta
std::list<int> second(4,100);//4 100
std::list<int> fourth(third);//kopia
\end{lstlisting}
    \end{column}
  \end{columns}
\end{frame}

\begin{frame}[fragile]
  \frametitle{list}
  \framesubtitle{Dostęp}
  \begin{itemize}
    \item front() - pierwszy element
    \item back() - ostatni element
  \end{itemize}
  \begin{lstlisting}
#include <iostream>
#include <list>

int main ()
{
  std::list<int> mylist;

  mylist.push_back(77);
  mylist.push_back(22);

  // now front equals 77, and back 22

  mylist.front() -= mylist.back();

  std::cout << "mylist.front() is now " << mylist.front() << '\n';

  return 0;
}
  \end{lstlisting}
\end{frame}

\begin{frame}[fragile]
  \frametitle{list}
  \framesubtitle{Iteratory}
  \begin{itemize}
    \item Bidirectional do typu
    \item \textit{begin()} - iterator do pierwszego elementu
    \item \textit{end()} - iterator za ostatni element
  \end{itemize}
  \begin{lstlisting}
#include <iostream>
#include <list>

int main ()
{
  int myints[] = {75,23,65,42,13};
  std::list<int> mylist (myints,myints+5);

  std::cout << "mylist contains:";
  for (std::list<int>::iterator it=mylist.begin(); it != mylist.end(); ++it)
    std::cout << ' ' << *it;

  std::cout << '\n';

  return 0;
}
\end{lstlisting}
\end{frame}

\begin{frame}[fragile]
  \frametitle{list}
  \framesubtitle{Rozmiar}
  \begin{itemize} 
    \item \textit{size} - bieżący rozmiar,
    \item \textit{max size} - rozmiar maksymalny
    \item \textit{empty} - test czy pusty
  \end{itemize}
  \begin{lstlisting}
#include <iostream>
#include <list>
int main (){
  std::list<int> mylist;
  std::cout << "0. size: " << mylist.size() << '\n';

  for (int i=0; i<10; i++) mylist.push_back(i);
  std::cout << "size: " << mylist.size() << '\n';
  int sum (0);
  while (!mylist.empty())
  {
     sum += mylist.front();
     mylist.pop_front();
     std::cout << "size: " << mylist.size() << '\n';
  }
  return 0;
}
\end{lstlisting}
\end{frame}

\begin{frame}[fragile]
  \frametitle{list}
  \framesubtitle{Zmiany}
  \begin{itemize}
    \item \textit{push front} i \textit{pop front} - dodaj / usuń na początku - Stały
    \item \textit{push back} i \textit{pop back} - dodaj / usuń na końcu - Stały
    \item \textit{insert} - wstawia element - liniowy z liczbą elementów
    \item \textit{erase} - usuwa elementy - liniowy z liczbą elementów
    \item \textit{swap} - wymienia zawartość z innym. Stały
    \item \textit{resize} - zmienia rozmiar
    \item \textit{clear} - czyści zawartość.
  \end{itemize}
  \begin{lstlisting}
#include <iostream>
#include <list>
int main (){
  std::list<int> mylist (2,100);// two ints with a value of 100
  mylist.push_front (200);
  mylist.push_front (300);
  std::cout << "mylist contains:";
  for (std::list<int>::iterator it=mylist.begin(); it!=mylist.end(); ++it)
    std::cout << ' ' << *it;
  std::cout << '\n';
  return 0;
}
\end{lstlisting}
\end{frame}

\subsection{pair}
\begin{frame}[fragile]
  \frametitle{pair}
  \framesubtitle{}
  \begin{columns}
    \begin{column}{0.5\textwidth}
    \begin{itemize}
      \item Przechowują unikalne elementy
      \item Zawsze posortowane
      \item Element jest jednocześnie kluczem
      \item Tylko jeden element o danej wartości
    \end{itemize}
    \begin{lstlisting}
template <class T1, class T2>
struct pair;
    \end{lstlisting}
    \end{column}
    \begin{column}{0.7\textwidth}
\begin{lstlisting}
#include <utility> // std::pair, std::make_pair
#include <string>
#include <iostream>
int main () {
  std::pair<std::string,double> product1;
  std::pair<std::string,double> product2 ("tomatoes",2.30);
  std::pair<std::string,double> product3 (product2);
  product1 = std::make_pair(std::string("lightbulbs"),0.99);
  product2.first = "shoes";                  // the type of first is string
  product2.second = 39.90;                   // the type of second is double
  std::cout << "The price of " << product1.first << " is $" << product1.second << '\n';
  std::cout << "The price of " << product2.first << " is $" << product2.second << '\n';
  std::cout << "The price of " << product3.first << " is $" << product3.second << '\n';
  return 0;
}
\end{lstlisting}
    \end{column}
  \end{columns}
\end{frame}

\subsection{set}

\begin{frame}[fragile]
  \frametitle{set}
  \framesubtitle{}
  \begin{columns}
    \begin{column}{0.4\textwidth}
    \begin{itemize}
      \item Przechowują unikalne elementy
      \item Zawsze posortowane
      \item Element jest jednocześnie kluczem
      \item Tylko jeden element o danej wartości
    \end{itemize}
    \end{column}
    \begin{column}{0.7\textwidth}
\begin{lstlisting}
#include <iostream>
#include <set>
bool fncomp (int lhs, int rhs) {return lhs<rhs;}
struct clcomp {
  bool operator() (const int& lhs, const int& rhs) const {return lhs<rhs;}
};

int main ()
{
  std::set<int> first;//pusty
  int myints[]= {10,20,30,40,50};
  std::set<int> second (myints,myints+5);
  std::set<int> third (second);//kopia
  std::set<int> fourth (second.begin(), second.end());  // iterator ctor.
  std::set<int,clcomp> fifth;                 // class as Compare
  bool(*fn_pt)(int,int) = fncomp;
  std::set<int,bool(*)(int,int)> sixth (fn_pt);  // function pointer as Compare
  return 0;
}
\end{lstlisting}
    \end{column}
  \end{columns}
\end{frame}

\begin{frame}[fragile]
  \frametitle{set}
  \framesubtitle{Iteratory}
  \begin{itemize}
    \item Bidirectional do typu
    \item \textit{begin()} - iterator do pierwszego elementu
    \item \textit{end()} - iterator za ostatni element
  \end{itemize}
  \begin{lstlisting}
#include <iostream>
#include <set>

int main ()
{
  int myints[] = {75,23,65,42,13};
  std::set<int> myset (myints,myints+5);

  std::cout << "myset contains:";
  for (std::set<int>::iterator it=myset.begin(); it!=myset.end(); ++it)
    std::cout << ' ' << *it;

  std::cout << '\n';

  return 0;
}
\end{lstlisting}
\end{frame}

\begin{frame}[fragile]
  \frametitle{set}
  \framesubtitle{Rozmiar}
  \begin{itemize} 
    \item \textit{size} - bieżący rozmiar,
    \item \textit{max size} - rozmiar maksymalny
    \item \textit{empty} - test czy pusty
  \end{itemize}
  \begin{lstlisting}
#include <iostream>
#include <set>

int main ()
{
  std::set<int> myints;
  std::cout << "0. size: " << myints.size() << '\n';

  for (int i=0; i<10; ++i) myints.insert(i);
  std::cout << "1. size: " << myints.size() << '\n';

  myints.insert (100);
  std::cout << "2. size: " << myints.size() << '\n';

  myints.erase(5);
  std::cout << "3. size: " << myints.size() << '\n';

  return 0;
}
\end{lstlisting}
\end{frame}

\begin{frame}[fragile]
  \frametitle{set}
  \framesubtitle{Zmiany}
  \begin{columns}
    \begin{column}{0.7\textwidth}
\begin{lstlisting}
#include <iostream>
#include <set>
int main (){
  std::set<int> myset;
  std::set<int>::iterator it;
  std::pair<std::set<int>::iterator,bool> ret;
  // set some initial values:10 20 30 40 50
  for (int i=1; i<=5; ++i) myset.insert(i*10);
  ret = myset.insert(20);
  if (ret.second==false)
    it=ret.first;  // "it" now points 20
  myset.insert (it,25);  // max efficiency
  myset.insert (it,24);  // max efficiency
  myset.insert (it,26);  // no max efficiency
  int myints[]= {5,10,15};// 10 already in set
  myset.insert (myints,myints+3);
  std::cout << "myset contains:";
  for (it=myset.begin(); it!=myset.end(); ++it)
    std::cout << ' ' << *it;
  std::cout << '\n';
  return 0;
}
\end{lstlisting}
    \end{column}
    \begin{column}{0.4\textwidth}
  \begin{itemize}
    \item \textit{insert} - wstawia element - logarytmiczny, lub stały
    \item \textit{erase} - usuwa elementy - logarytmiczny, stały lub liniowy
    \item \textit{swap} - wymienia zawartość z innym. Stały
    \item \textit{clear} - czyści zawartość.
  \end{itemize}
    \end{column}
  \end{columns}

\end{frame}

\subsection{map}

\begin{frame}[fragile]
  \frametitle{map}
  \framesubtitle{}
  \begin{columns}
    \begin{column}{0.4\textwidth}
    \begin{itemize}
      \item Przechowują pary (klucz, wartość), klucz jest unikalny
      \item Zawsze posortowane po kluczu
      \item podobne do set
      \item Dostęp przez [] lub find jest obarczony kosztem logarytmicznym
    \end{itemize}
    \end{column}
    \begin{column}{0.7\textwidth}
\begin{lstlisting}
#include <iostream>
#include <map>
bool fncomp (char lhs, char rhs) {return lhs<rhs;}
struct classcomp {
  bool operator() (const char& lhs, const char& rhs) const
  {return lhs<rhs;}
};
int main ()
{
  std::map<char,int> first;
  first['a']=10;
  first['b']=30;
  first['c']=50;
  first['d']=70;
  std::map<char,int> second (first.begin(),first.end());
  std::map<char,int> third (second);
  std::map<char,int,classcomp> fourth;                 // class as Compare
  bool(*fn_pt)(char,char) = fncomp;
  std::map<char,int,bool(*)(char,char)> fifth (fn_pt); // function pointer as Compare
  return 0;
}
\end{lstlisting}
    \end{column}
  \end{columns}
\end{frame}

\begin{frame}[fragile]
  \frametitle{map}
  \framesubtitle{Iteratory}
  \begin{itemize}
    \item Bidirectional do typu
    \item \textit{begin()} - iterator do pierwszego elementu
    \item \textit{end()} - iterator za ostatni element
  \end{itemize}
  \begin{lstlisting}
#include <iostream>
#include <map>
int main (){
  std::map<char,int> mymap;

  mymap['b'] = 100;
  mymap['a'] = 200;
  mymap['c'] = 300;
  // show content:
  for (std::map<char,int>::iterator it=mymap.begin(); it!=mymap.end(); ++it)
    std::cout << it->first << " => " << it->second << '\n';
  return 0;
}
\end{lstlisting}
\end{frame}

\begin{frame}[fragile]
  \frametitle{map}
  \framesubtitle{Rozmiar}
  \begin{itemize} 
    \item \textit{size} - bieżący rozmiar,
    \item \textit{max size} - rozmiar maksymalny
    \item \textit{empty} - test czy pusty
  \end{itemize}
  \begin{lstlisting}
#include <iostream>
#include <map>

int main ()
{
  std::map<char,int> mymap;

  mymap['a']=10;
  mymap['b']=20;
  mymap['c']=30;

  while (!mymap.empty())
  {
    std::cout << mymap.begin()->first << " => " << mymap.begin()->second << '\n';
    mymap.erase(mymap.begin());
  }

  return 0;
}
\end{lstlisting}
\end{frame}

\begin{frame}[fragile]
  \frametitle{map}
  \framesubtitle{Dostęp}
  \begin{columns}
    \begin{column}{0.7\textwidth}
\begin{lstlisting}
#include <iostream>
#include <set>
int main (){
  std::set<int> myset;
  std::set<int>::iterator it;
  std::pair<std::set<int>::iterator,bool> ret;
  // set some initial values:10 20 30 40 50
  for (int i=1; i<=5; ++i) myset.insert(i*10);
  ret = myset.insert(20);
  if (ret.second==false)
    it=ret.first;  // "it" now points 20
  myset.insert (it,25);  // max efficiency
  myset.insert (it,24);  // max efficiency
  myset.insert (it,26);  // no max efficiency
  int myints[]= {5,10,15};// 10 already in set
  myset.insert (myints,myints+3);
  std::cout << "myset contains:";
  for (it=myset.begin(); it!=myset.end(); ++it)
    std::cout << ' ' << *it;
  std::cout << '\n';
  return 0;
}
\end{lstlisting}
    \end{column}
    \begin{column}{0.4\textwidth}
  \begin{itemize}
    \item \textit{insert} - wstawia element - logarytmiczny, lub stały, zwraca parę (iterator, bool)
    \item \textit{find} - znajduje element, zwraca iterator do lub .end() - logarytmiczny
    \item \textit{[]} - jak find, logarytmiczny
    \item \textit{clear} - czyści zawartość.
  \end{itemize}
    \end{column}
  \end{columns}

\end{frame}

\section{Algorytmy}

\begin{frame}[fragile]
  \frametitle{for each}
    \begin{itemize} 
    \item Stosuje funkcję do wszystkich elementów z zakresu
  \end{itemize}
  \begin{lstlisting}
#include <iostream>     // std::cout
#include <algorithm>    // std::for_each
#include <vector>       // std::vector

void myfunction (int i) {  // function:
  std::cout << ' ' << i;
}

struct myclass {           // function object type:
  void operator() (int i) {std::cout << ' ' << i;}
} myobject;

int main () {
  std::vector<int> myvector;
  myvector.push_back(10);
  myvector.push_back(20);
  myvector.push_back(30);

  std::cout << "myvector contains:";
  for_each (myvector.begin(), myvector.end(), myfunction);
  std::cout << '\n';

  // or:
  std::cout << "myvector contains:";
  for_each (myvector.begin(), myvector.end(), myobject);
  std::cout << '\n';

  return 0;
}
  \end{lstlisting}
\end{frame}

\begin{frame}[fragile]
  \frametitle{find}
  \begin{itemize} 
    \item Zwraca iterator do pierwszego wystąpienia elementu
  \end{itemize}
  \begin{lstlisting}
// find example
#include <iostream>     // std::cout
#include <algorithm>    // std::find
#include <vector>       // std::vector

int main () {
  // using std::find with array and pointer:
  int myints[] = { 10, 20, 30, 40 };
  int * p;

  p = std::find (myints, myints+4, 30);
  if (p != myints+4)
    std::cout << "Element found in myints: " << *p << '\n';
  else
    std::cout << "Element not found in myints\n";

  // using std::find with vector and iterator:
  std::vector<int> myvector (myints,myints+4);
  std::vector<int>::iterator it;

  it = find (myvector.begin(), myvector.end(), 30);
  if (it != myvector.end())
    std::cout << "Element found in myvector: " << *it << '\n';
  else
    std::cout << "Element not found in myvector\n";

  return 0;
}
  \end{lstlisting}
\end{frame}

\begin{frame}[fragile]
  \frametitle{fill}
  \begin{itemize} 
    \item Wypełnia wartością
  \end{itemize}
  \begin{lstlisting}
// fill algorithm example
#include <iostream>     // std::cout
#include <algorithm>    // std::fill
#include <vector>       // std::vector

int main () {
  std::vector<int> myvector (8);// myvector: 0 0 0 0 0 0 0 0

  std::fill (myvector.begin(),myvector.begin()+4,5);   // myvector: 5 5 5 5 0 0 0 0
  std::fill (myvector.begin()+3,myvector.end()-2,8);   // myvector: 5 5 5 8 8 8 0 0

  std::cout << "myvector contains:";
  for (std::vector<int>::iterator it=myvector.begin(); it!=myvector.end(); ++it)
    std::cout << ' ' << *it;
  std::cout << '\n';

  return 0;
}
  \end{lstlisting}
\end{frame}

\begin{frame}[fragile]
  \frametitle{sort}
  \begin{itemize} 
    \item Sortuje rosnąco
  \end{itemize}
  \begin{lstlisting}
// sort algorithm example
#include <iostream>     // std::cout
#include <algorithm>    // std::sort
#include <vector>       // std::vector
bool myfunction (int i,int j) { return (i<j); }
struct myclass {
  bool operator() (int i,int j) { return (i<j);}
} myobject;
int main () {
  int myints[] = {32,71,12,45,26,80,53,33};
  std::vector<int> myvector (myints, myints+8); 
  // using default comparison (operator <):
  std::sort (myvector.begin(), myvector.begin()+4);
  // using function as comp
  std::sort (myvector.begin()+4, myvector.end(), myfunction);
  // using object as comp
  std::sort (myvector.begin(), myvector.end(), myobject);
  // print out content:
  std::cout << "myvector contains:";
  for (std::vector<int>::iterator it=myvector.begin(); it!=myvector.end(); ++it)
    std::cout << ' ' << *it;
  std::cout << '\n';
  return 0;
}
  \end{lstlisting}
\end{frame}

\begin{frame}
	\frametitle{unique}
	Działa na posortowanej kolekcji. Powtarzające elementy zostaję przesunięte na koniec kolekcji,
	zwraca iterator za ostatni unikalny element.
\end{frame}

\subsection{Binary search}

\begin{frame}[fragile]
  \frametitle{lower bound}
  \begin{itemize} 
    \item Zwraca iterator do elementu nie mniejszego niż argument
    \item Zakres musi być posortowany
    \item koszt logarytmiczny
  \end{itemize}
  \begin{lstlisting}
// lower_bound/upper_bound example
#include <iostream>     // std::cout
#include <algorithm>    // std::lower_bound, std::upper_bound, std::sort
#include <vector>       // std::vector

int main () {
  int myints[] = {10,20,30,30,20,10,10,20};
  std::vector<int> v(myints,myints+8);           // 10 20 30 30 20 10 10 20
  std::sort (v.begin(), v.end());                // 10 10 10 20 20 20 30 30
  std::vector<int>::iterator low,up;
  low=std::lower_bound (v.begin(), v.end(), 20); //          ^
  up= std::upper_bound (v.begin(), v.end(), 20); //                   ^
  std::cout << "lower_bound at position " << (low- v.begin()) << '\n';
  std::cout << "upper_bound at position " << (up - v.begin()) << '\n';
  return 0;
}
  \end{lstlisting}
\end{frame}

\begin{frame}[fragile]
  \frametitle{binary search}
  \begin{itemize} 
    \item Zwraca \textit{true} gdy jest i \textit{false} gdy nie
    \item Kolekcja musi być posortowana
    \item Koszt logarytmiczny
  \end{itemize}
  \begin{lstlisting}
// binary_search example
#include <iostream>     // std::cout
#include <algorithm>    // std::binary_search, std::sort
#include <vector>       // std::vector
bool myfunction (int i,int j) { return (i<j); }
int main () {
  int myints[] = {1,2,3,4,5,4,3,2,1};
  std::vector<int> v(myints,myints+9);                         // 1 2 3 4 5 4 3 2 1
  // using default comparison:
  std::sort (v.begin(), v.end());
  std::cout << "looking for a 3... ";
  if (std::binary_search (v.begin(), v.end(), 3))
    std::cout << "found!\n"; else std::cout << "not found.\n";
  // using myfunction as comp:
  std::sort (v.begin(), v.end(), myfunction);
  std::cout << "looking for a 6... ";
  if (std::binary_search (v.begin(), v.end(), 6, myfunction))
    std::cout << "found!\n"; else std::cout << "not found.\n";
  return 0;
}
  \end{lstlisting}
\end{frame}

\end{document}
