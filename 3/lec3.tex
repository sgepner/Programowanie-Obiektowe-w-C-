%\documentclass[draft,10pt]{beamer}
\documentclass[10pt]{beamer}

\usepackage[english,polish]{babel}
\usepackage[utf8]{inputenc}
\usepackage{polski}

\usepackage{define}

\eventtitle{Programowanie obiektowe w języku C++}
\title{Programowanie obiektowe w języku C++}
\author[shortname]{Stanis{\l}aw Gepner}
\institute[shortinst]{sgepner@meil.pw.edu.pl}
\date{}

\setbeamertemplate{blocks}[rounded][shadow=true]
\setbeamertemplate{navigation symbols}[]

\begin{document}

\frame{
    \titlepage
}

\section{Wspomnienia}

\begin{frame}
  \frametitle{Co już wiemy}
  
  \centering
  \begin{enumerate}
    \item Klasy i struktury
    \item modyfikatory \textit{public} i \textit{protected}
    \item Metody - funkcje klasy
    \item Konstruktory i destruktor klasy
  \end{enumerate}
\end{frame}

\section{static}

\begin{frame}[fragile]
  \frametitle{Zmienne statyczne}
  \begin{columns}
    \begin{column}{0.48\textwidth}
\vspace{-0.2cm}
\begin{lstlisting}
class foobar{
  public:
    static int a;
    int b = 5; // -std=c++11
};
int main(){
  foobar f1, f2;
  ...
}
\end{lstlisting}
    \end{column}
    \begin{column}{0.48\textwidth}
      \begin{itemize}
        \item Nowe słówko \textit{static}
        \item Współdzielone przez wszystkie instancje klasy
        \item Inicjalizacja poza ciałem bez słówka \textit{static}
        \item Można użyć bez instancji klasy!
        \item Można zmienić wartość
        \item Przykład:
      \end{itemize}
    \end{column}
  \end{columns}
\end{frame}

\begin{frame}[fragile]
  \frametitle{Zmienne statyczne const ...}
  \begin{columns}
    \begin{column}{0.48\textwidth}
\vspace{-0.2cm}
\begin{lstlisting}
class foobar{
  public:
    static const int a;
    int b = 5; // -std=c++11
};
int main(){
  foobar f1, f2;
  ...
}
\end{lstlisting}
    \end{column}
    \begin{column}{0.48\textwidth}
      \begin{itemize}
        \item \textit{static const}
        \item Współdzielone przez wszystkie instancje klasy
        \item Inicjalizacja w ciele
        \item Można użyć bez instancji klasy!
        \item Nie można zmienić wartości
        \item Przykład:
      \end{itemize}
    \end{column}
  \end{columns}
\end{frame}

\begin{frame}[fragile]
  \frametitle{Funkcje statyczne}
  \begin{columns}
    \begin{column}{0.48\textwidth}
\vspace{-0.2cm}
\begin{lstlisting}
#include <iostream>
using namespace std;
class foobar{
  public:
    static int a;
    int b;
    
    static void static_do(int new_d)
    {
      cout << "a=" << a << endl;
      a = new_d;
      //b = a; // nie wolno
    }
};
int foobar::a = 9;
int main(){
  foobar f1, f2, f3;
  cout << f1.a << endl;
  foobar::static_do(9);
  cout << f3.a << endl;
}
\end{lstlisting}
    \end{column}
    \begin{column}{0.48\textwidth}
      \begin{itemize}
        \item \textit{static typ nazwa()\{\}}
        \item Współdzielone przez wszystkie instancje klasy
        \item Można użyć bez instancji klasy!
        \item Może działać tylko na zmiennych, które są \textit{static}
        \item Przykład:
      \end{itemize}
    \end{column}
  \end{columns}
\end{frame}

\section{Operatory}

\begin{frame}[fragile]
  \frametitle{Operatory}
      \begin{itemize}
        \item Operatorem może być symbol, np.: +, -, *, /
        \item albo \textit{new new[] delete []delete}
        \item albo teź konwersja typów (przemilczeć)
        \item W C++ użycie operatora jest związane z wywołaniem metody
        \item C++ pozwala na definiowanie własnych operatorów
        \item ... ale nie dla typów wbudowanych
        \item Definicje mogą być w ciele, albo poza
        \item Operatory mogą być jedno (+a) i dwu argumentowe (a+b)
        \item W zasadzie metoda może być dowolna (co nie znaczy, że powinna)
        \item Operator powinien wykonywać przewidywalną operację np: \\
        c=a+b nie powinien zmieniać argumentów a, b (chociaz własciwie to czemu nie?)
        \item Jest tego dużo
      \end{itemize}
\end{frame}

\begin{frame}[fragile]
  \frametitle{Operatory}
  \framesubtitle{Lista niepełna}
          + operator dodawania, może być jedno lub dwuargumentowy\\
          - operator odejmowania, może być jedno lub dwuargumentowy\\
          * operator mnożenia, dwuargumentowy\\
          / operator dzielenia, dwuargumentowy\\
          \% operator dwuargumentowy\\
          \& jedno lub dwu argumentowy\\
          = operator przypisania, może być domyślny\\
$\sim$, !, =, 
$<$, $>$
+=, -=
*=, /=
\%=, \&=
|=, <<
>>, >>=
<<=, ==
!=, $<$=
$>$=, \&\&
||, ++
--, ,, -$>$*, -$>$, [], \\
() - wywołania - ile chcemy argumentowy\\
poniższe operatory gdy ich nie zdefiniujemy robi to za nas kompilator
new, new[], delete, delete[]

\end{frame}

\begin{frame}[fragile]
  \frametitle{Operator}
  \framesubtitle{Na razie bez sensu}
  \begin{columns}
    \begin{column}{0.48\textwidth}
\begin{lstlisting}
//Ogolna skladnia
typ operator@ (argumenty)
{
// operacje 
}

//wywolanie
operator@(args);
\end{lstlisting}
    \end{column}
    \begin{column}{0.48\textwidth}
    \begin{lstlisting}
#include <iostream>
using namespace std;
class foo{
  public:
  int a;
};
void operator+(foo a) //zle
{
  a.a*=-1;
  cout << "Wywolano operator + a=" << a.a << endl;
}
void operator*(foo& a)//tez zle
{
  a.a*=-1;
  cout << "Wywolano operator - a=" << a.a << endl;
}
int main(){
  foo f1;
  f1.a = 9;
  cout << f1.a << endl;
  operator*(f1); //to tez jest wywolanie
  +f1;
  *f1;
  cout << f1.a << endl;
}
\end{lstlisting}
    \end{column}
  \end{columns}
\end{frame}

\begin{frame}[fragile]
  \frametitle{Operator dwuargumentowy}
  \framesubtitle{Wciąż bez sensu}

\begin{lstlisting}
#include <iostream>
using namespace std;
class foo{
  public:
  int a;
};
void operator+(foo& a, foo& b) //nie zwraca??
{
  a.a = a.a + b.a;
  cout << "Wywolano operator + a=" << a.a << endl;
}

int main(){
  foo f1, f2;
  f1.a = 9;
  f2.a = 3;
  cout << f1.a << endl;
  f1+f2;
  cout << f1.a << endl;
}
\end{lstlisting}
Na razie trochę bez sensu, więcej przykładów ...
\end{frame}

\begin{frame}[fragile]
  \frametitle{Operator =}
  \framesubtitle{Teraz lepiej}
  
  \begin{columns}
    \begin{column}{0.48\textwidth}
\begin{lstlisting}
class Foo {
public:
...
  Foo & operator=(const Foo &prawa);
...
}

Foo& Foo::operator=(const Foo &p)
{
  if (this != &p)// Samo kopia?
  {
   ... // Wykonaj
  }
  return *this; //zwroc siebie
}

Foo a, b;
...
b = a;   // Jak b.operator=(a);
\end{lstlisting}
    \end{column}
    \begin{column}{0.48\textwidth}
      \begin{itemize}
        \item Musi być w ciele klasy (?)
        \item Argument (prawa strona) jest \textit{const}
        \item Operator zwraca by umożliwić a=b=c
        \item Samo przypisanie? a=a\\
        \textit{if (this != \&rhs)}
        \item przykład ...
      \end{itemize}
    \end{column}
  \end{columns}
  
\end{frame}

\begin{frame}[fragile]
  \frametitle{Operator +=, -=, *=}
  \framesubtitle{Teraz lepiej}
  
  \begin{columns}
    \begin{column}{0.48\textwidth}
\begin{lstlisting}
class Foo {
public:
...
  Foo & operator+=(const Foo &prawa);
...
}

Foo& Foo::operator+=(const Foo &p)
{
  if (this != &p)// a+=a?
  {
   ... // Wykonaj
  }
  return *this; //zwroc siebie
}

Foo a, b;
...
b += a;   // Jak b.operator+=(a);
\end{lstlisting}
    \end{column}
    \begin{column}{0.48\textwidth}
      \begin{itemize}
        \item W ciele lub poza klasy
        \item Argument (prawa strona) jest \textit{const}
        \item Operator zwraca by umożliwić a+=b+=c ?
        \item przykład ...
      \end{itemize}
    \end{column}
  \end{columns}
  
\end{frame}

\begin{frame}[fragile]
  \frametitle{Operator dwuargumentowy}
  \framesubtitle{W ciele - niebezpiecznie}

    \begin{lstlisting}
#include <iostream>
using namespace std;
class Foo{
  public:
  Foo operator+(const Foo &prawy) const {
    Foo res = *this;  // Kopia res(*this);
    res += prawy;     // Uzyj +=
    return res;       // !
  }
};

int main(){
  Foo a, b, c;
  
  c = a+b;
\end{lstlisting}
\end{frame}

\begin{frame}[fragile]
  \frametitle{Operator dwuargumentowy}
  \framesubtitle{Poza ciałem - \textit{friend}}

    \begin{lstlisting}
#include <iostream>
using namespace std;
class foo{
  public:
  foo(int a) : a(a) {}
  void print() { cout << a << endl; }
  private:
  int a;
  friend foo operator+(const foo& a, const foo& b);
};
foo operator+(foo& a, foo& b)
{
  foo tmp = a;
  tmp.a += b.a;
  return tmp;
}

int main(){
  foo f1(3), f2(2);
  f1.print();
  f2.print();
  foo f3 = f1+f2;
  f3.print();
}
\end{lstlisting}
\end{frame}

\subsection{friend}

\begin{frame}
  \frametitle{friend}
  
  
  \begin{itemize}
    \item \textit{friend}
    \item Ma dostęp do atrybutów \textit{private} i \textit{protected}
    \item Pozwala na pelniejszą hermetyzację
    \item To co musiało by być \textit{public} może zostać udostępnione tylko pewnym klasom
    \item Zwiekszona zależność pomiędzy klasami
    \item Przyjaźń jest jednostronna, deklarowana przez klasę dopuszczającą
    \item Nieistotne gdzie w ciele
    \item Można do wielu
    \item Klasy jak i funkcje
    \item Nie jest przechodnia
  \end{itemize}
\end{frame}

\begin{frame}[fragile]
  \frametitle{<<}
  
\begin{lstlisting}
#include <iostream>
using namespace std;
class foo{
  public:
  foo(int a, int b) : a(a), b(b) {}
  private:
  int a;
  int b;
  friend std::ostream& operator<<(std::ostream& os, const foo& f);
};

std::ostream& operator<<(std::ostream& os, const foo& f)
{
    os << "[" << f.a << ", " << f.b << "]";
    return os;
}

int main()
{
  foo f1(3,4), f2(5,6);
  std::cout << f1 << std::endl;
  std::cout << f2 << std::endl;
}
\end{lstlisting}
\end{frame}

\begin{frame}[fragile]
  \frametitle{Obiekt funkcyjny ()}
  \framesubtitle{funktor}
  
\begin{itemize}
  \item Obiekt, który mozna wywołać jak funkcję
  \item Konieczne zdefiniowanie operatora \textit{()}
  \item Zaletą nad funkcją jest to, że funktor jest pełnoprawnym obiektem
  \item ... może więc mieć określony stan
  \item ... konstruktor, destruktor, itd.
\end{itemize}

\begin{lstlisting}
#include <iostream>
 
class foo {
  public:
    foo (int x) : _x( x ) {}
    int operator() (int y) { return _x + y; }
    int operator() (int a, int b, int c) {return _x + a + b - c;}
  private:
    int _x;
};
 
int main(){
  foo f1(5);
  foo f2(4);
  std::cout << f1( 6 ) << endl;
  std::cout << f2( 6, 7, 2 ) << endl;
  return 0;
}
\end{lstlisting}

\end{frame}

\section{Dziedziczenie - początek}

\begin{frame}[fragile]
  \frametitle{Dziedziczenie}
  \begin{itemize}
    \item Gdy kolejna klasa rozszerza możliwości już istniejącej
    \item Pozwala tworzyć nowe w oparciu o stare
    \item Współdzielenie kodu
    \item Interpretowanie obiektu tak jakby był obiektem z ktorego dziedziczy
    \item Hierarchia klas
  \end{itemize}
  
  \begin{lstlisting}
class nazwa :[operator_widocznosci] nazwa_klasy_bazowej //na razie
{
  //definicja_klasy
};
\end{lstlisting}
\end{frame}
















\end{document}
